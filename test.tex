\documentclass[12pt]{article}

\usepackage{graphicx}
\usepackage{amsmath,amsthm,amsfonts}                % AMS Math
\usepackage{thmtools}                                       % Theorem Tools
\usepackage{bm}                                             % Bold Math

\title{LaTex test kram}
\author{Testautor}
\date{April 2022}

\begin{document}

\maketitle

% Start vom Fließtext
\pagebreak
\section{Einleitung}
Das ist ein test Text. Hier steht eine Formel direkt im Paragraphen: 
$\sum_{n = 1}^{\infty} x^n$. Hier steht dann wiederum eine Formel einzeln als Figure:
\begin{align}
  \int_{0}^{\infty} 6x^4-\frac{4}{6} \,dx \label{random integral}
\end{align}
Eine Formel kann man anhand ihres Labels referenzieren, dann passt sich
die Zahl automatisch an. Z.B. das integral (\ref{random integral}) wird automatisch
erkannt. \linebreak

% Tony Bild
\begin{figure}[h]
  \centering
  \includegraphics[width=0.25\textwidth]{Bilder/Unvernichtet.jpg}
  \caption{Ein schönes Bild}
  \label{fig:mesh1}
\end{figure}

% Liste
\begin{itemize}
  \item Das ist ein Listenpunkt!
  \item Das ist ein weiterer Listenpunkt!!
\end{itemize}
% Ein Enter ist egal, wird gedruckt als wäre es in einer Zeile
Das steht alles in
einer Zeile

%Eine leere Zeile startet einen neuen Paragraphen
Das ist ein neuer Paragraph.
\pagebreak

Die Tabelle \ref{table:data} ist eine vollformatierte Tabelle. Toll!
\begin{table}[h!]
\centering
\begin{tabular}{||c c c c||} 
 \hline
 Col1 & Col2 & Col2 & Col3 \\ [0.5ex] 
 \hline\hline
 1 & 6 & 87837 & 787 \\ 
 2 & 7 & 78 & 5415 \\
 3 & 545 & 778 & 7507 \\
 4 & 545 & 18744 & 7560 \\
 5 & 88 & 788 & 6344 \\ [1ex] 
 \hline
\end{tabular}
\caption{Eine tolle Tabelle}
\label{table:data}
\end{table}
\end{document}